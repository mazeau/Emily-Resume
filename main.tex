% LaTeX file for resume 
% This file uses the resume document class (res.cls)

\documentclass[10pt, line]{res} 

% \usepackage{fontspec} % font types
% \setmainfont{Times New Roman}

\usepackage{enumitem}
\setitemize[0]{leftmargin=*, itemsep=0pt}

\usepackage{times} 
\setlist[itemize]{topsep=4pt,noitemsep}
%\usepackage{newcent}   % uses new century schoolbook postscript font 
\usepackage[margin=0.5in]{geometry}%left=0.75in,right=0.75in,bottom=0.5in]{geometry}
\resumewidth=7.5in
\newsectionwidth{0pt}  % So the text is not indented under section headings
\usepackage{fancyhdr}  % use this package to get a 2 line header
\usepackage{textgreek}
\renewcommand{\headrulewidth}{24pt} % suppress line drawn by default by fancyhdr
\setlength{\headheight}{0pt} % allow room for 2-line header
\setlength{\headsep}{0pt}  % space between header and text
\setlength{\headheight}{24pt} % allow room for 2-line header
\pagestyle{fancy}     % set pagestyle for document
\rhead{ {\it E. Mazeau}\\{\it p. \thepage} } % put text in header (right side)
\cfoot{}                                     % the foot is empty
\topmargin=-0.75in % start text higher on the page

\usepackage[bookmarks,
pdftitle={Resume},
pdfauthor={Emily Mazeau},
pdfsubject={},
pdfcreator={},
pdfproducer={},
pdfkeywords={},
pdfpagemode={},
pdfstartview=FitH,
colorlinks=true,
linkcolor=black,   %Color for normal internal links.
anchorcolor=magenta, %Color for anchor text.
citecolor=green,   %Color for bibliographical citations in text.
filecolor=black, %Color for URLs which open local files.
menucolor=red,     %Color for Acrobat menu items.
urlcolor=black]{hyperref}

\begin{document}
\thispagestyle{empty} % this page has no header  
\name{\Large{Emily J. Mazeau}}

\address{127 Fox Point Road, Newington, NH 03801\\\href{mailto:mazeau.e@northeastern.edu}{mazeau.e@northeastern.edu}\\(603) 502-9457\\\href{https://www.github.com/mazeau}{github.com/mazeau}}
\begin{resume}

\vspace{-10pt}
%%%%%%%%%%%%%%%%%%%%%%%%%%%%%%%%%%%%%%%%%%%%%

% My undergrad research focused on engineering co-culture systems for the degradation of cellulose. I decided I wanted to learn something new for my PhD and focused on computational chemistry, modeling, and the development of the RMG software for heterogeneous catalysis. I am looking to stay in research and development in either computational chemistry, software engineering, data science, or machine learning/AI jobs and to continue publishing.

%%%%%%%%%%%%%%%%%%%%%%%%%%%%%%%%%%%%%%%%%%%%%

\section{\centerline{\large{Skills}}}

\textbf{Data Science:} Pandas, statistical modeling, scikit-learn, NumPy, SciPy, RegEx, SQL, Mathematica/Wolfram language\\
\textbf{Programming:} Git, Python, C++, Cython, MATLAB, Unix/Linux environment, SLURM, HPC, PyCharm, Bash\\
\textbf{Computational Chemistry:} Cantera, Chemkin, ASE, RMG, RDKit, AutoTST, Schrödinger, LAMMPS, VASP, NWChem\\
\textbf{Professional:} Social Chair for Oak Ridge Postdoctoral Association, Co-Chair for Session in the Nuclear Engineering Division at AIChE
\textbf{Security Clearance:} In process for Q/TS//RD
% \textbf{Laboratory:} Cell culture, HPLC, 
% \textbf{Programming \& Simulation:} Python, MatLab, Java, C++, Quantum chemistry, Cantera, Chemkin \\
% \textbf{Computer:} Microsoft Office, Adobe Creative Suite \\
% \textbf{Laboratory:} Cell culture, protein biochemistry, bioreactor operations and maintenance

%%%%%%%%%%%%%%%%%%%%%%%%%%%%%%%%%%%%%%%%%%%%%
\section{\centerline{\large{Experience}}}

\textbf{Postdoctoral Research Associate}, \href{https://www.ornl.gov/nnd}{\sl Computational Chemistry}, \href{https://www.ornl.gov/}{ORNL}, Oak Ridge, TN \hspace{0.2in}  \hfill Nov 2022 - Present
\begin{itemize}
    \item Modeled complex actinide systems using molecular dynamics and LAMMPS
    \item Calculated DFT for actinides using VASP and NWChem
    \item contributed to open-source software
    % \item Social Chair for the Oak Ridge Postdoctoral Association
\end{itemize}
\vspace{-5pt}

\textbf{Graduate Researcher}, \href{http://www.northeastern.edu/comocheng/people/}{\sl CoMoChEng Lab, Northeastern University}, Boston, MA \hspace{0.2in}  \hfill Sept 2017 - May 2022
\begin{itemize}
    \item Focused on detailed kinetic models and quantum chemistry in collaboration with the Combustion Research Facility at Sandia National Lab
    \item Collaborated with Mitsubishi Heavy Industries (MHI) to study the hydroxylammonium nitrate (NH$_3$OHNO$_3$) decomposition mechanism
    \item Developed the RMG software for heterogeneous catalysis and added linear scaling relations, metal catalysts, complex surface reactions, and coverage dependence
    \item Used RMG to accurately predicted the catalytic partial oxidation of methane on Pt(111) and Rh(111) and 81 novel metal catalysts to discover catalysts that have the highest selectivity for partial oxidation 
    \item Used RMG to model the catalytic dimerization and oligomerization of ethylene to 1-butene and successfully predicted experimental product distributions
    \item Mentored undergraduate students conducting independent research projects
\end{itemize}
\vspace{-5pt}

\textbf{Undergraduate Researcher}, \href{https://homepages.rpi.edu/~koffam/}{\sl Koffas \& Collins Labs, Rensselaer Polytechnic Institute}, Troy, NY  \hfill Nov 2015 - May 2017
\begin{itemize}
    \item Focused on producing violacein, a natural antibiotic with anti-tumor properties, using a cellulose degradation system
    % \item Optimized expression of violacein in \textit{E. coli} by testing different media and expression conditions
    \item Expression optimization of  EGI1, an enzyme that performs the first step of cellulose degradation
    % \item Tested the effect of media conditions, temperatures, and induction point by assessing protein secretion and activity
    \item Engineered and cloned a signal peptide library for an enzyme, Cel9AT, that performs cellulose degradation into glucose
    % \item Tested activity of the library to identify the signal peptide with highest cellulose degradation activity
    \item Taught and mentored undergraduate and high school students conducting independent research projects
\end{itemize}
\vspace{-5pt}

% \textbf{Undergraduate Researcher}, \href{https://homepages.rpi.edu/~koffam/}{\sl Collins Lab, Rensselaer Polytechnic Institute}, Troy, NY  \hfill May 2016 - Aug 2016
% \begin{itemize}
%     \item Expression optimization of  EGI1, an enzyme that performs the first step of cellulose degradation
%     \item Tested the effect of media conditions, temperatures, and induction point by assessing protein secretion and activity
% \end{itemize}
% \vspace{-5pt}

% \textbf{Undergraduate Researcher}, \href{https://homepages.rpi.edu/~koffam/}{\sl Koffas Lab, Rensselaer Polytechnic Institute}, Troy, NY  \hfill Nov 2015 - May 2016
% \begin{itemize}
%     \item Engineered and cloned a signal peptide library for an enzyme, Cel9AT, that performs cellulose degradation into glucose
%     \item Tested activity of the library to identify the signal peptide with highest cellulose degradation activity
% \end{itemize}
% \vspace{-5pt}

%%%%%%%%%%%%%%%%%%%%%%%%%%%%%%%%%%%%%%%%%%%%%

\section{\centerline{\large{Education}}} 

\href{http://www.northeastern.edu/}{\textbf{Northeastern University}}, Boston, MA \hspace{0.2in}  \hfill Sept 2017 - May 2022 \\
Ph.D., Chemical Engineering - {\sl Thesis title:} Ongoing Developments in Automatic Generation of Microkinetic Models for Heterogeneous Catalysis using RMG 
% \vspace{-5pt}

\href{http://www.rpi.edu/}{\textbf{Rensselaer Polytechnic University}}, Troy, NY \hspace{0.2in}  \hfill Sept 2014 - May 2017 \\
B.S., Chemical and Biological Engineering

%%%%%%%%%%%%%%%%%%%%%%%%%%%%%%%%%%%%%%%%%%%%%

\section{\centerline{\large{Publications}}}

\begin{itemize}
    \item \href{https://doi.org/10.1021/acs.jcim.2c00965}{The RMG Database for Chemical Property Prediction.
    \textit{J. Chem. Inf. Model.} 2022 \textit{62} (20), 4906-4915.}
    \item \href{https://doi.org/10.1021/acs.iecr.1c03076}{Extensive High-Accuracy Thermochemistry and Group Additivity Values for Halocarbon Combustion Modeling. \textit{Ind. Eng. Chem. Res.} 2021 \textit{60} (43), 15492-15501.}
    \item \href{https://pubs.acs.org/doi/10.1021/jacsau.1c00276} {Quantifying the Impact of Parametric Uncertainty on Automatic Mechanism Generation for CO$_2$ Hydrogenation on Ni(111). \textit{JACS Au} 2021 \textit{1} (10), 1656-1673.}
    \item \href{https://doi.org/10.1021/acscatal.0c04100}{Automated Mechanism Generation Using Linear Scaling Relationships and Sensitivity Analyses Applied to Catalytic Partial Oxidation of Methane. \textit{ACS Catal.} 2021 \textit{11} (12), 7114-7125.}
    \item \href{https://doi.org/10.1021/acs.jcim.0c01480}{Reaction Mechanism Generator v3.0.0: Advances in Automatic Mechanism Generation. \textit{J Chem. Inf. Model.} 2021 \textit{61} (6), 2686-2696.}
    \item \href{https://doi.org/10.1021/acs.iecr.9b01464}{Computer-generated kinetics for coupled heterogeneous/homogeneous systems: a case study in catalytic combustion on platinum. \textit{Ind. Eng. Chem. Res.} 2019 \textit{58} (38), 17682-17691.}
    \item \href{https://doi.org/10.1021/acssynbio.8b00186}{Engineering \textit{Bacillus megaterium} strains to secrete cellulases for synergistic cellulose degradation in a microbial community. \textit{ACS Synth. Biol.} 2018 \textit{7} (10), 2413-2422.}
\end{itemize}
% \href{https://pubs.acs.org/doi/abs/10.1021/acs.jpca.7b07361}{\textbf{Bhoorasingh, P. L.}, Slakman, B. L., Seyedzadeh Khanshan, F., Cain, J. Y., \& West, R. H. Automated transition state theory calculations for high-throughput kinetics. \textit{J. Phys. Chem. A} 121 (37), pp 6896-6904 (2017).}

% \href{http://xlink.rsc.org/?DOI=c5cp04706d}{\textbf{Bhoorasingh, P. L.} \& West, R. H. Transition state geometry prediction using molecular group contributions. \textit{Phys. Chem. Chem. Phys.} 17, 32173-32182 (2015).}

% Van de Vijver, R., Vandewiele, N. M., \textbf{Bhoorasingh, P. L.}, Slakman, B. L., Seyedzadeh Khanshan, F., Carstensen, H. H., Reyniers, M. F., Marin, G. B., West, R. H. \& Van Geem, K. M. (2015). Automatic Mechanism and Kinetic Model Generation for Gas‐and Solution‐Phase Processes: A Perspective on Best Practices, Recent Advances, and Future Challenges. \textit{Int. J. Chem. Kinet.}, 47(4), 199-231.

%%%%%%%%%%%%%%%%%%%%%%%%%%%%%%%%%%%%%%%%%%%%%

% \section{\centerline{\large{Memberships}}}
% % {\sl NEU Chem. Eng. Graduate Student Council} \hspace{0.2in}  \hfill Webmaster (2012 - 2014) \\
% {\sl MIT Caribbean Club} \hspace{0.2in}  \hfill Secretary (2004-05), Treasurer (2005-06), VP (2006-07) \\
% {\sl Jamaica U20 Water Polo} \hspace{0.2in}  \hfill Vice Captain (2002-2004)
% {\sl MIT Water Polo} \hspace{0.2in}  \hfill 2003-2004
 
\end{resume} 
\end{document}














